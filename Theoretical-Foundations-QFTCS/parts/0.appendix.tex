\appendix

\chapter{Fock space}

The symmetric Fock space is the Hilbert space used in the construction of our quantum field theory in curved spacetimes for a scalar field. However, it must be constructed from a existent Hilbert space, that physically can have the interpretation of the space of one particle.

Let \(\mcH\) be a Hilbert space, then the Fock space associated to it, \(\mcF(\mcH)\), is defined as
\begin{equation}
    \mcF(\mcH)\equiv\oplus_{n=0}^{\infty}\left(\otimes^{n}\mcH\right).
\end{equation}

Therefore, an element \(\Psi\in\mcF(\mcH)\) is infinite dimension vector whose each entry is a tensor product of elements of \(\mcH\), i.e.,
\begin{equation}
    \Psi=\left(\psi,\psi^a,\psi^{ab},\dots\right),
\end{equation}
where
\begin{equation}
    \psi^{a_1\dots a_j}\in\otimes^j\mcH.
\end{equation}

The symmetric Fock space, \(\mcF_S(\mcH)\), is defined as
\begin{equation}
    \mcF_S(\mcH)\equiv\oplus_{n=0}^{\infty}\left(\otimes^{n}_S\mcH\right),
\end{equation}
that is very similar to the ordinary Fock space, however, the possible entries must be symmetric, i.e.,
\begin{equation}
    \psi^{a_1\dots a_j}=\psi^{(a_1\dots a_j)}.
\end{equation}

In order to the Fock space be a Hilbert space, we equip it with inner product that act in two elements \(\Psi,\Phi\in\mcF_S(\mcH)\) as
\begin{equation}
    \langle\Psi,\Phi\rangle_{\mcF_S}\equiv\overline{\psi}\phi+\overline{\psi}_a\phi^a+\dots,
\end{equation}
in which the elements \(\overline{\psi}_a\) are the dual vectors correspondent to the conjugated one in \(\overline{\mcH}\).

\subsection*{Algebra of operators}
Now, we define the annihilation and creation operators in the Fock space. Let \(\xi^a\in\mcH\), then the annihilation operator \(a(\overline{\xi}):\mcF_S(\mcH)\to\mcF_S(\mcH)\) is defined as
\begin{equation}
    a(\overline{\xi})\Psi\equiv\left(\overline{\xi}_{a_1}\psi^{a_1},\sqrt{2}\overline{\xi}_{a_1}\psi^{a_1a_2},\sqrt{3}\overline{\xi}_{a_1}\psi^{a_1a_2a_3},\dots\right).
\end{equation}

Similarly, the creation operator \(a^{\dagger}(\xi):\mcF_S(\mcH)\to\mcF_S(\mcH)\) is defined as
\begin{equation}
    a^{\dagger}(\xi)\Psi\equiv\left(0,\psi\xi^a,\sqrt(2)\xi^{(a_1}\psi^{a_2)},\sqrt{3}\xi^{(a_1}\psi^{a_2a_3)},\dots\right).
\end{equation}

\begin{proposition}\label{prop:ccr}
    Let \(\xi,\eta\in\mcH\), then the annihilation and creation operators has commutation relation given by
    \begin{equation}
        [a(\overline{\xi}),a(\overline{\eta})]=[a^{\dagger}(\xi),a^{\dagger}(\eta)]=0
    \end{equation}
    and
    \begin{equation}
        [a(\overline{\xi}),a^{\dagger}(\eta)]=\overline{\xi}_a\eta^a\mathbb{I}.
    \end{equation}
\end{proposition}
\begin{proof}
    We shall restric our proof to the first two non-trivial termos. Using our definitions, we have
    \begin{equation}
        a(\overline{\xi})a(\overline{\eta})\Psi=\left(\sqrt{2}\overline{\xi}_{a_2}\overline{\eta}_{a_1}\psi^{a_1a_2},\sqrt{6}\overline{\xi}_{a_2}\overline{\eta}_{a_1}\psi^{a_1a_2a_3},\dots\right).
    \end{equation}

    Similarly, we have
    \begin{equation}
        a(\overline{\eta})a(\overline{\xi})\Psi=\left(\sqrt{2}\overline{\eta}_{a_2}\overline{\xi}_{a_1}\psi_2^{a_1a_2},\sqrt{6}\overline{\eta}_{a_2}\overline{\xi}_{a_1}\psi_3^{a_1a_2a_3},\dots\right).
    \end{equation}

    However, from the Fock space symmetry,
    \begin{equation}
        \overline{\xi}_{a_2}\overline{\eta}_{a_1}\psi_2^{a_1a_2}=\overline{\eta}_{a_2}\overline{\xi}_{a_1}\psi_2^{a_1a_2}\;\;\;\text{and}\;\;\;\overline{\xi}_{a_2}\overline{\eta}_{a_1}\psi_3^{a_1a_2a_3}=\overline{\eta}_{a_2}\overline{\xi}_{a_1}\psi_3^{a_1a_2a_3}.
    \end{equation}

    Arguments exploring the symmetry also holds for the other components. Since it is applied in an arbitrary vector \(\Psi\), is evident that the commutator vanishes. 
    Similarly, from the definition of creation operator, we have
    \begin{equation}
        \begin{aligned}
            &a^{\dagger}(\xi)a^{\dagger}(\eta)\Psi=\left(0,0,\sqrt{2}\psi\xi^{(a_1}\eta^{a_2)},\sqrt{6}\xi^{(a_1}\eta^{a_2}\psi^{a_3)},\dots\right)\\
            &a^{\dagger}(\eta)a^{\dagger}(\xi)\Psi=\left(0,0,\sqrt{2}\psi\eta^{(a_1}\xi^{a_2)},\sqrt{6}\eta^{(a_1}\xi^{a_2}\psi^{a_3)},\dots\right)
        \end{aligned}.
    \end{equation}

    Again, invoking the symmmetry of the Fock space, we have
    \begin{equation}
        \psi_0\xi^{(a_1}\eta^{a_2)}=\psi_0\eta^{(a_1}\xi^{a_2)}\;\;\;\text{and}\;\;\;\xi^{(a_1}\eta^{a_2}\psi_1^{a_3)}=\eta^{(a_1}\xi^{a_2}\psi_1^{a_3)}.
    \end{equation}

    With the same strategies for the other components, we find that the commutator of creation operators are null.

    Finally, the elements for the last commutator are
    \begin{subequations}
        \begin{align}
            &a(\overline{\xi})a^{\dagger}(\eta)\Psi=\left(\psi\overline{\xi}_a\eta^a,2\overline{\xi}_a\eta^{(a}\psi^{a_2)},3\overline{\xi}_a\eta^{(a}\psi^{a_1a_2),\dots}\right)\\
            &a^{\dagger}(\eta)a(\overline{\xi})\Psi=\left(0,\overline{\xi}_a\psi^a\eta^{a_1},2\overline{\xi}_a\psi^{a(a_1}\eta^{a_2)},\dots\right).
        \end{align}
    \end{subequations}

    Notice that
    \begin{equation}
        2\overline{\xi}_a\eta^{(a}\psi^{a_1)}=\overline{\xi}_a\eta^a\psi^{a_1}+\overline{\xi}_a\psi^a\eta^{a_1}
    \end{equation}
    and
    \begin{subequations}
        \begin{align}
            3\overline{\xi}_{a_1}\eta^{(a_1}\psi^{a_2a_3)}&=\frac{1}{2}\overline{\xi}_a[\eta^{a_1}\psi^{a_2a_3}+\eta^{a_1}\psi^{a_3a_2}+\eta^{a_2}\psi^{a_1a_3}\\
            &+\eta^{a_2}\psi^{a_3a_1}+\eta^{a_3}\psi^{a_1a_2}+\eta^{a_3}\psi^{a_2a_1}]\\
            &=\overline{\xi}_{a_1}\eta^{a_1}\psi^{(a_2a_3)}+\overline{\xi}_{a_1}\left[\psi^{a_1a_3}\eta^{a_2}+\psi^{a_1a_2}\eta^{a_3}\right]\\
            &=\overline{\xi}_{a_1}\eta^{a_1}\psi_2^{(a_2a_3)}+2\overline{\xi}_{a_1}\psi_2^{(a_1a_2}\eta^{a_3)}.
        \end{align}
    \end{subequations}

    Thus, if we procede with similar manipulations for the other components, we find that
    \begin{equation}
        [a(\overline{\xi}),a^{\dagger}(\eta)]\Psi=\overline{\xi}_a\eta^a\Psi,
    \end{equation}
    that proves the desired relations.
\end{proof}